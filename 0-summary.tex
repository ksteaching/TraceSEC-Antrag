%!TEX root = Beschreibung_des_Vorhabens.tex

\subparagraph*{Motivation and Context:}
Software is often unnecessarily insecure. While developers tend to be sensitive to security issues in highly sensitive domains, developers in ordinary software projects are often unfamiliar with security aspects. For example, the correct implementation of security-critical functionalities REFfiresmith2003 REFtondel2008 is not always ensured. There are only a few security experts available who cannot assist in all projects. Guidance for developers is limited and often not tailored to a specific context. Security must be considered from the beginning of the development and throughout all subsequent phases and activities. Both individual developer and the organization at large should increase their security competency for facing new attacks and challenges. Both individual and organizational learning are treated separately and, thus, is often neglected. There is little time and inclination for external training, as the transfer from training examples to the real application context is a hurdle. Since development, problem solving, and continuous learning are not integrated, they do not benefit from each other. Currently, security-specific techniques focus on security alone. They are poorly integrated with ordinary development tools and practices. Furthermore, security relevant information is scattered all around the system. 

\subparagraph*{Research Vision:}
We envision using and extending well-known concepts such as quality models and trace to capture and refine security-related requirements, and to follow the development process through all stages of develop­ment. In parallel, a software development organization needs to learn from its previous security problems and solutions. Changing knowledge in the environment, such as new attack patterns, need to affect policies and tools. We envision an innovative approach for integrating three core activities:

\begin{enumerate}
    \item The development of software which is potentially security-relevant.
    \item Diagnosis and analysis of problems and attacks within a cycle of continuous improvement. 
    \item Use, reuse, and exploitation of security knowledge and insight in a socio-technical way.
\end{enumerate}

All levels and phases of software development must be covered, including informal early requirements and prioritization, as well as formalization and heuristic checking of security properties with respect to the initial informal demands. 

We want to investigate the potential of extending quality models and combining them with an exten­ded and adapted tracing technique. Quality models and specific artifacts are tied together by tracing development activities at a very fine-grained level. Those same tools (extended quality model and tracing) will be reused for all three core activities. Individual and organizational learning will benefit from concrete and contextualized examples that are generated as by-product of the two other tasks. These types of learning are different from machine learning.

\subparagraph*{Key Contributions:}
We extend both quality models and tracing in order to cover informal and formal activities in a socio-technical environment. This scope includes developers, customers, but also code and monitoring. Development, problem analysis, and learning must be adjusted to support each other.

The term quality model is used for generic collections and relationships of quality aspects, such as reliability, usability, etc. in ISO 25 010. In a project, a quality model must be refined to provide concrete and measurable specifications of relevant quality attributes, such as security. We trace the refinement of informal and generic quality requirements into more specific interpretations (e.g. from performance to run-time performance of a module). This will require manual entries and automated monitoring. Extended quality models will cover the entire spectrum from informal to formal aspect, but only relevant aspects should be included, which will be difficult to decide. Extended tracing will connect human considerations and requirements on security as well as technical security decisions. We will exploit specific relationships in the security process to develop heuristics for identifying relevant parts, and for reducing effort. 

Added value is expected from reuse among the three above-mentioned activities of development, problem analysis, and learning from concrete examples. We plan to integrate our approach into standard/waterfall-based and agile processes and investigate how and to which degree the intended methodology can serve those types of process.

\subparagraph*{Challenges to be Overcome:}
It will be challenging to identify what is relevant in the informal, formal, heuristic, and measurable facets of security. There are people and artifacts involved, and in principle, their interaction can be followed and traced. To make this concept feasible, the overarching opportunity and challenge will be to reuse results from one of the above-mentioned core activities (1, 2, 3) to boost the others. This is a key concept in this proposal.

Planned Validation: A first step in evaluating the feasibility of the TraceSEC approach will show whether results from one activity can be adapted and reused to facilitate the other activities. We will provide a set of examples and use cases to demonstrate how manual and automated steps are interrelated. We will point out and investigate how an extended quality model and extended traces can be built as a by-product of software development activities, and how educational material can be derived from these real-world development activities. To validate the approach, we will complement feasibility demonstrations by within-subject studies in which we empirically study how traditional develop­ment activities perform compared to those supported by TraceSEC. After the experimental comparison, we will interview subjects for their preference and perceived differences. We will triangulate perceived problems with objective indicators collected in the experiment. Long-term effects are difficult to study; we will use a case study to observe what happens to quality model and traces over several months.

\subparagraph*{Research Impact and Measure of Success:} The case below is inspired by a real-world case. It is used in this proposal to illustrate our concepts and will later be used for demonstrating the feasibility and benefit of our approach. This case provides a context to instantiate all aspects of formal and informal, as well as all relevant activities (development, problem analysis, learning). In particular, it illustrates the envisioned techniques for reusing results across these activities: How can a similarity measure between problem cases help to find an older case as an inspiration to solve a new problem? How can we identify recommended next steps to a novice developer, based on earlier successful cases? How can educational material (e.g. screen videos) be generated from traces and connected artifacts? These questions will guide research, and our results will be measured with respect to those guiding questions. We intend to publish in both software engineering and security venues (conferences and journals). Specific human-centered software engineering communities will show the interdisciplinary impact of our work.

\paragraph*{A Realistic Case as Running Example}
In many projects, security issues can have dramatic consequences. For example, in December 2019 a bug in a router led to 20,000 patient records to be publicly available~\cite{ct2019a}. One of the main security issues was insufficient access control for the patient data. There was no internal access control implemented, and everyone who was in the network of the doctor’s office had access to all patient data. Due to the bug in the router, there were open ports which allowed everyone beyond the office to get into the network from the outside and to access the patient data. 

For discussion purposes, we assume a fictive software company that is going to implement a secure management system for patient data in this example. 

At first, they are collecting requirements the system has to meet. Besides considering classical functional and non-functional requirements, domain specific requirements from relevant standards must be observed, such as IEC 62304 and IEC 62366 for medical device software. Customer requirements must be checked for security implications, e.g. by automatically recommending and including relevant standards. As the project is dealing with personal data, implementing an appropriate access control is an example of such a security-implied requirement.

In system design, the architecture has to address all requirements captured. As a consequence, the required access control has to be reflected in the systems architecture. In the above-mentioned real system, access control for preventing outside attackers had been considered (including a router with firewall), but no access control for inside attackers had been planned. Such missing or neglected security mechanisms, as well as flawed and insecure security mechanisms, should be automatically detected at design time. This is a case of heuristic identification of conceptual or design problems during development. It requires a representation, that is formal enough for automated search.

The next pitfall for our fictive company occurs during implementation. The company skips the implementation of a (previously required) security mechanism to save time and effort. In addition, an unexperienced developer again selects an outdated and unsecure mechanism, e.g., a deprecated cryptographic hash function. Automated support for developers monitoring source code for security issues could warn, providing relevant documentation and best practices but also in generating at least stubs for the required security mechanisms. With respect to our TraceSEC concepts, it would be important to follow and trace security-related requirements through all steps and artifacts for identifying glitches like this. If a security-related design concept like access control has no trace to an adequate part of the implementation, a heuristic warning can be issues.

Configuration of the system is yet another important part. In addition to internal access control, the company also implemented a limitation of accesses per hour to harden the system against brute force attacks. However, the pre-configured number of accesses per hour might lead to a denial of service if the software is used without adaption to the context – e.g. a big hospital instead of a small doctor’s office.

At run-time, changes in the security assumptions will occur. For example, it has been recently shown that SHA1 has become an insecure hash algorithm due to new attack knowledge. New requirements may also be stimulated by unexpected observations at run-time, like several hundred access attempts from a single IP-address which leads to all accesses being blocked by the above-mentioned access limit. Requiring dynamic IP-address filters could be a decision by the human experts. This account of technical problems and solution attempts should be retrieved in a future case with similar profile for informing problem analysis. We also foresee the opportunity to use a sequence of decisions, consequences and artifacts, as a real-case contextualized example for developers who try to learn more about security within the scope of their particular tasks and company.

\paragraph*{Different perspectives and experience in collaborating}
The applicants and their research groups approach the topic from different angles: Jan Jürjens and his team has a long record of security modeling (e.g., in UMLsec [12]) and formal techniques to verify REF and ensure security REF. Kurt Schneider and his group focus on socio-technical aspects of software engineering, such as elicitation and validation of requirements, knowledge management [26] or explainability [5]. They build tools to support the collaboration of technology and humans in a software engineering context (e.g., FOCUS [21] [24], AppForMeetingsREF, HeRA [27]). The collaboration of both groups started in 2008, and has been intensified over the years: From publications on detecting security-related requirements [11] to SecVolution in the DFG Priority Programm 1593  [8] [35] [33]. The successful cooperation demonstrates and strengthened the ability to join their very different competencies in the area of security within software engineering. This encourages the applicants to now go far beyond the previous joint work and address a topic at the intersection of informal and formal, as well as technical and socio-technical issues. The key concept of combing that with individual and organizational learning benefits from the rare case where diverse skills and viewpoints have reached maturity in collaboration [9]. This encourages us to jointly address the challenges of TraceSEC and reach results none of the groups could reach by itself.
