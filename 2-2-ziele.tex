%!TEX root = Beschreibung_des_Vorhabens.tex
\todo[inline]{Rahmen}
In the TraceSEC project, we want to develop a semi-automated socio-technical method. It will combine existing and new security-related elements as a source for use and reuse in three core activities:
\begin{enumerate}
	\vspace{-0.5em} \item development of software that could be security-relevant, 
	\vspace{-0.5em} \item problem diagnosis and analysis of security problems (vulnerabilities or process flaws), and 
	\vspace{-0.5em} \item for continuous learning of an organization and its developers.
\end{enumerate}

The method needs to be socio-technical since it has to consider and optimize human needs and abilities together with new techniques for automating essential tasks. Those tasks include tracing of activities, extraction of relevant steps, and formalization for identifying similar cases. Weaving traces with connected artifacts and parts of the quality model allows to identify meaningful scenarios. Finding and delivering scenarios similar to a current task at hand assist developers in all three core activities: (1) Development, (2) problem analysis, and (3) learning. 

We want to develop techniques and demonstrators illustrating the potential of adding value to scenarios, e.g., by turning screenshots and a sequence of steps into a security training video. Since it relies on real cases from that same environment, the problem of de-contextualization and irrelevance is addressed and mitigated: Contextualization comes for free when the scenarios come from the same context anyway. Relevance is likely, due to the fact that those cases were relevant in the past.

The method will, therefore, include a number of exemplary generators and matching algorithms or classifiers. They operate on traces, with artifacts and parts of a quality model attached. The method will integrate automated and semi-automated parts into software-developing organizations to increase their competency and efficiency in dealing with security-related issues, yielding the following benefits:

\begin{itemize}
\vspace{-0.5em} \item These organizations will recognize security implications on all levels represented by the quality model.
\vspace{-0.5em} \item They will widen the scope of security activities to informal (human-based) and formal (code and reasoning) aspects.
\vspace{-0.5em} \item This will allow companies to see and maintain relationships and dependencies between security-related elements and activities. They will reflect and reason about complete scenarios rather than focusing on single security challenges in isolation.
\vspace{-0.5em} \item Pattern matching will identify similar threats and scenarios. Clustering scenario traces and generating compact pieces of knowledge will make them relevant for the task at hand.
\vspace{-0.5em} \item Since we consider traces and scenarios first-class objects, they will be treated as knowledge items and reused in all three core activities, transformed to the respective purpose. This helps to make acquired knowledge relevant to the tasks at hand REF.
\vspace{-0.5em} \item The value of traces will be weighed against the effort to create them. Observing activities and generating traces will reduce effort, while mechanisms for comparing and reusing them will add value (see WP 3).
\end{itemize}

We suggest referring to an extended quality model as reference for security-related information and activities. The method does not rely on a certain development type (such as agile, traditional, hybrid), but can be configured to a given type. We seek to investigate the basic principles that can be mapped to various environments and development styles.
